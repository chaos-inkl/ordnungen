\documentclass[a4paper, 12pt]{scrartcl}

%% PDF SETUP
\usepackage[pdftex, bookmarks, colorlinks, breaklinks,
pdfusetitle,plainpages=false]{hyperref}
\hypersetup{linkcolor=blue,pdfauthor={Chaos inKL. e.V.},citecolor=blue,filecolor=black,urlcolor=blue,plainpages=false}
\usepackage[utf8]{inputenc}
\usepackage[T1]{fontenc}
\usepackage[ngerman]{babel}

\usepackage{ccicons}
\usepackage{url}

% Use URW Garamond No. 8 as a default font. (getnonfreefonts)
\usepackage[urw-garamond]{mathdesign}
%\renewcommand{\rmdefault}{ugm}
% Optima as a sans serif font.
\renewcommand*\sfdefault{uop}
\usepackage[protrusion=true,expansion=true]{microtype}
% Recalculate page setup based on new font.
\KOMAoptions{DIV=last}
\usepackage{todonotes}
\pagestyle{plain}

\renewcommand*\thesection{\S~\arabic{section}}

\begin{document}
\title{Beitragsordnung}
\subtitle{des Vereins "Chaos inKL. e.V."}
\author{}
\date{}

\maketitle

\noindent Gemäß §6 der Satzung des Vereins "`Chaos inKL. e.V."'
haben Mitglieder die von der Mitgliederversammlung
festgelegten Beiträge zu entrichten. Das Nähere bestimmt die
nachstehende, von der Mitgliederversammlung am 10.01.2012 mit Wirkung ab
dem 10.01.2012 beschlossene Beitragsordnung. \\[0.5cm]
Geändert gemäß dem Beschluss der Mitgliederversammlung vom 14. Dezember 2013.

\section{Aufnahmebeitrag}
\begin{enumerate}
	\item Der Aufnahmebeitrag wird bei Aufnahme in den Verein fällig. Er ist nach positivem Bescheid über den Aufnahmeantrag mit dem ersten Mitgliedsbeitrag zu zahlen.
	\item Der Aufnahmebeitrag beträgt EUR 10,-.
	\item Die Mitgliedschaft beginnt, sobald der Aufnahmeantrag angenommen wurde und Aufnahmegebühr und Mitgliedsbeitrag bezahlt wurden.
\end{enumerate}

\section{Mitgliedsbeitrag}
\begin{enumerate}
	\item Doppelmitglieder entrichten den in \ref{dmbeitrag}.\ref{dmbeitragbetrag} festgelegten Beitrag.
	\item Die Mitglieder des Vereins, welche keine Doppelmitgliedschaft gewählt haben, haben einen monatlichen
    Mitgliedsbeitrag von EUR 30,- zu entrichten.
	\item Der Mitgliedsbeitrag ist monatlich zu entrichten. Er kann jährlich oder halbjährlich im Voraus entrichtet werden.
	\item Der Beitrag ist binnen vier Wochen nach Annahme des Aufnahmeantrages und folgend jeweils zum dritten Werktag eines Monats unaufgefordert auf das Vereinskonto einzuzahlen. Bei jährlicher Vorauszahlungen ist der Beitrag zum dritten Werktag des Januars, bei halbjährlicher Vorauszahlung jeweils zum dritten Werktag im Januar und im Juli zu entrichten.
	\item Das Entrichten des Mitgliedsbeitrags kann entweder durch Lastschrift erfolgen, oder durch unaufgefordertes Einzahlen auf das Vereinskonto. In begründeten Ausnahmefällen kann mit dem Schatzmeister eine Barzahlung vereinbart werden. In diesem Fall müssen die Beiträge halbjährlich oder jährlich entrichtet werden.
	\item Im Falle nicht fristgerechter Entrichtung der Beiträge ruht die Mitgliedschaft. Des weiteren wird für jede schriftliche Mahnung eine Bearbeitungspauschale von EUR 5,- erhoben.
	\item Die Mitglieder halten ihre Mitgliedsdaten aktuell. Vorzugsweise
    werden Änderungen an die E-Mailadresse
    vorstand@chaos-inkl.de mitgeteilt. 
\end{enumerate}

\section{Doppelmitgliedschaft}\label{dm}
Der Verein "`Chaos inKL. e.V."' bietet seinen Mitgliedern die Option, eine Doppelmitgliedschaft mit dem Chaos Computer Club e.V. (CCC e.V.) einzugehen. Der Verein "`Chaos inKL. e.V."' übernimmt die Organisation der Mitgliedschaft, die sonst dem Mitglied obliegen würde.

\subsection{Beiträge}\label{dmbeitrag}
\begin{enumerate}
	\item \label{dmbeitragbetrag}Für Mitglieder, die die Doppelmitgliedschaft wählen, beträgt der monatliche Beitrag für den Verein "`Chaos inKL. e.V."' EUR 30,-.
	\item Mitgliedsbeitrag und Aufnahmegebühr des CCC e.V.\ richten sich nach dessen jeweils gültiger Beitragsordnung.
\end{enumerate}

\section{Organisation}
\begin{enumerate}
	\item Der Aufnahmebeitrag und der Mitgliedsbeitrag für Chaos inKL.\ e.V.\ m"ussen separat "uberwiesen werden.
	\item Im Falle einer Doppelmitgliedschaft muss der Mitgliedsbeitrag für den CCC e.V.\ ebenfalls separat "uberwiesen werden.
	\item Doppelmitgliedschaft:
\begin{enumerate}
	\item Mitglieder, die die Doppelmitgliedschaft wählen, sind Mitglieder beider Vereine mit sämtlichen Rechten und Pflichten.
	\item Doppelmitglieder bezahlen die Mitgliedsbeiträge für beide Vereine an den Verein "`Chaos inKL. e.V."'. Dieser leitet den Beitrag an den CCC e.V. weiter.
	\item Der Verein "`Chaos inKL. e.V."' übermittelt die für die Mitgliederverwaltung notwendigen personenbezogenen Daten (Name, Anschrift, ggf. Nachweis für Ermäßigung) an den CCC e.V.
	\item Endet die Mitgliedschaft im Verein "`Chaos inKL. e.V."', besteht die Mitgliedschaft im CCC e.V. unabhängig weiter. In diesem Fall muss die Mitgliedschaft im CCC e.V. durch das Mitglied selbstständig weiterorganisiert werden.
	\item Die Mitgliedschaft im Verein "`Chaos inKL. e.V."' ist unabhängig von der Mitgliedschaft im CCC e.V.
\end{enumerate}
\end{enumerate}

\section{Ausnahmen}
Auf Antrag kann der Vorstand in begründeten Ausnahmefällen für einzelne Mitglieder Ausnahmen von dieser Beitragsordnung beschließen.

\vfill

\begin{flushright}
	\ccby \\
	{\small
		Dieses Dokument steht unter der Creative Commons Namensnennung 3.0 Deutschland Lizenz. Mehr Informationen zur Lizenz unter \url{https://creativecommons.org/licenses/by/3.0/de/}
	}
\end{flushright}

\end{document}
